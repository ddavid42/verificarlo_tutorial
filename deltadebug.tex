\section{Pinpointing bugs with Delta-Debug: Archimedes method}

\begin{figure}[h]
  \centering
  \begin{tikzpicture}[scale=1.5]
    \draw [magenta] (0,0) circle (1);
    \draw [fill, black] (0,0) circle (.02);
    \begin{scope}[yshift=1cm]
    \draw [blue, turtle={ home,
      right=90, forward=.5773502691896257, % tan(pi/6)
      right=60, forward=2*.5773502691896257,
      right=60, forward=2*.5773502691896257,
      right=60, forward=2*.5773502691896257,
      right=60, forward=2*.5773502691896257,
      right=60, forward=2*.5773502691896257,
      right=60, forward=.5773502691896257
    }];
    \end{scope}
\end{tikzpicture}
  \caption{Archimedes method to approximate PI with a 6-sided circumscribed polygon.
    \label{fig:archimedes}
  }
\end{figure}

In this section we demonstrate how we can use Verificarlo to precisely localize a numerical bug in a program. The localization method is based on the Zeller's Delta-Debug reduction method~\cite{zeller2001automated}. Verificarlo uses the Interflop's stochastic Delta-Debug library by Bruno Lathuilière and François Fevotte.

In 200BC Archimedes proposed the first numerical method for computing $\pi$.
Archimedes method uses one $6.n$-sided circumscribed polygon to the unit circle
whose area provides an upper bound for $\pi$ and one $6.n$-sided inscribed polygon
whose area provides a lower bound for $\pi$.

Here we will use the circumscribed polygon to approximate $\pi$.
Figure~\ref{fig:archimedes} shows a 6-sided circumscribed polygon to the unit
circle. Archimedes shows geometrically that the half-perimeter of the polygons (also converging to $\pi$) can be computed with the following recursive sequence,

\begin{align*}
  T_1 &= \frac{1}{\sqrt{3}} \\
  T_{i+1} &= \frac{\sqrt{T_i^2+1} - 1}{T_i} \\
  \frac{P_{i}}{2} &= 6 \times 2^{i-1} \times T_{i} \xrightarrow[i \to \infty]{} \pi
\end{align*}

In this part of the tutorial we will work inside the \texttt{archimedes/} folder.

\begin{question}
  \begin{enumerate}[(a)]
    \item Open the file \texttt{archimedes.c} and analyze the code.
    \item The provided makefile builds the program using Verificarlo. Run the program
      with the IEEE backend (\texttt{libinterflop\_ieee.so}) and with the MCA backend
      (\texttt{libinterflop\_mca.so -{}-precision 53}).
    \item How many digits are significant in the computed result?
  \end{enumerate}
\end{question}

The previous experiment shows that computation becomes numerically unstable
around iteration 15. Where is the error in the code? To help pinpointing the error,
we are going to use Delta-Debug.

Delta-Debug (DD) is a general bug reduction method that allows to efficiently find a
minimal set of conditions that triger a bug. In this case, we are going to consider
the set of floating-point instructions in the program. We are using DD to
find a minimal set of instructions responsible for the instability in the output.

\begin{table}[h]
  \centering
  \begin{tabular}{lcr}
    Step & Instructions with MCA noise & Numerically Stable \\
    \midrule
    1    & 1 2 3 4 . . . . & stable \\
    2    & . . . . 5 6 7 8 & unstable \\
    \midrule
    3    & . . . . 5 6 . . & stable \\
    4    & . . . . . . 7 8 & unstable \\
    \midrule
    5    & . . . . . . 7 . & unstable \\
    Result (ddmin) & . . . . . . 7 . & \\
  \end{tabular}
  \caption{Example of Delta-Debug bug minimization\label{tab:deltadebug}}
\end{table}

Table~\ref{tab:deltadebug} shows a simple DD execution to find a reduced
instruction set responsible for a numerical instability. By testing
instructions sub-sets and their complement, DD is able to find smaller failing
sets step by step. DD stops when it finds a failing set where it cannot remove
any instruction. In this case, DD is able to find a minimal failing set (ddmin)
of size 1 (which is therefore also minimum). There is no guarantee of unicity.

By default, Interflop's Delta-Debug implementation iterates to find all the
possible different ddmin sets. At the end, it produces the rddmin-cmp set which
is the complement of the union of the ddmin sets. The rddmin-cmp set therefore
includes the "stable" instructions and excludes the "unstable" instructions.

To use Delta-Debug, we need to write two scripts: \begin{itemize}
  \item A first script \texttt{ddRun <output\_dir>}, is responsible for running the program and writing its output inside the \texttt{<output\_dir>} folder.
  \item A second script \texttt{ddCmp <reference\_dir> <current\_dir>}, is passed two folders containing the outputs from a reference run and the current run. The \texttt{ddCmp} script must return with success when the deviation between both runs is acceptable, and fail if the deviation is unnaceptable.
\end{itemize}

\begin{question}
  \begin{enumerate}[(a)]
    \item Open the files \texttt{ddRun} and \texttt{ddCmp} and analyze how they work.
  \end{enumerate}
\end{question}

\texttt{ddRun} a \texttt{ddCmp} depend on the user's application and the error tolerance
of the application domain; therefore it is hard to provide a single script that fits all cases. That is why we require the user to manually write these scripts. Once the scripts
are written, to launch a Delta-Debug session we use the following command,

\begin{verbatim}
VFC_BACKENDS="libinterflop_mca.so -p 53 -m mca" vfc_ddebug ddRun ddCmp
\end{verbatim}

where \texttt{VFC\_BACKENDS} specifies the noise model that will be used to
simulate numerical errors. Here we provide a simple Makefile target that
runs this command,
\begin{verbatim}
make dd MCA_MODE="-p 53 -m mca"
\end{verbatim}

\begin{question}
  \begin{enumerate}[(a)]
    \item Open the file \texttt{Makefile} and analyze the \texttt{make dd} target.
    \item Run the \texttt{make dd} target.
  \end{enumerate}
\end{question}

Now that you have run Delta-Debug, you can see it has found two minimal failing sets (ddmin0 and ddmin1). The sets outputs can be found in \texttt{dd.line/ddmin0} and \texttt{dd.line/ddmin1}. To see the "unstable" instructions belonging to each set open the  \texttt{dd.line/ddmin{0,1}/dd.line.include} files. We can also find the union of "unstable" instructions in the \texttt{dd.line/rddmin-cmp/dd.line.exclude} file.

\begin{verbatim}
$ cat dd.line/ddmin0/dd.line.include
0x0000000000400e5c: archimedes at archimedes.c:16
$ cat dd.line/ddmin1/dd.line.include
0x0000000000400e89: archimedes at archimedes.c:17

# We can also get the union of culprit instructions with

$ cat dd.line/rddmin-cmp/dd.line.exclude
0x0000000000400e5c: archimedes at archimedes.c:16
0x0000000000400e89: archimedes at archimedes.c:17
\end{verbatim}

We observe that DD finds that instructions at line 16 and 17 are responsible for the output numerical unstability. To facilitate the debugging process we include a script that
transforms this output, into the error format used by compilers. Therefore we can use
a standard IDE to pinpoint culprit instructions inside the code.

\begin{question}
  \begin{enumerate}[(a)]
    \item Run the \texttt{make dderrors} target. This should open a Vim session.
    \item Type \texttt{:cw} to open the error quick-fix window. Use \texttt{:cn} and \texttt{:cp} to move to the next and previous "unstable" instructions.
  \end{enumerate}
\end{question}

Line 17 points to a substraction operation (doublesub) between $s$ and $1$. One can see that as $T_{i+1}$ (\texttt{tii}) becomes smaller, $s$ becomes closer to $1$. Therefore it seems line 17 could trigger a catastrophic cancellation.

To test this hypothesis we can use the pb mode with a high precision (such as
60).  PB mode simulates cancellations. Using a precision higher than 53 means
that only cancellations that cancel more than 7 bits will introduce errors.
Using such a high-precision means we only introduce noise for strong
cancellation but do not introduce noise in other kind of operations.

\begin{question}
  \begin{enumerate}[(a)]
    \item Run Delta-Debug with a PB noise model at precision 60.
      \begin{verbatim}
make dd MCA_MODE="-m pb -p 60"
      \end{verbatim}
    \item Run \texttt{make dderrors}. How many "unstable" operations are detected?
      Can you explain why line 16 was flagged as "unstable" in the previous run and not anymore?
  \end{enumerate}
\end{question}

To fix this problem, we can try to rewrite the culprit expression in line 17.
Observe that,

\begin{align*}
  T_{i+1} &= \frac{\sqrt{T_i^2+1} - 1}{T_i} = \frac{\sqrt{T_i^2+1} - 1}{T_i} \times \frac{\sqrt{T_i^2+1} + 1}{\sqrt{T_i^2+1} + 1} \\
          &= \frac{(T_i^2 + 1) - 1}{T_i(\sqrt{T_i^2+1} + 1)} \\
          &= \frac{T_i}{\sqrt{T_i^2+1} + 1}
\end{align*}

The new formula is interesting since it eliminates the substraction that triggered the cancellation.

\begin{question}
  \begin{enumerate}[(a)]
    \item Modify \texttt{archimedes.c} to use the previous expression rewriting.
    \item Study the numerical instability of the new version. Is the problem fixed?
  \end{enumerate}
\end{question}
